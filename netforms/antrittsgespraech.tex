\documentclass[NET,a4,12pt,ngerman]{netforms}

\usepackage[utf8]{inputenc}
\usepackage{tumlang}
\usepackage{tumcontact}
\usepackage{scrpage2}

\geometry{%
	top=10mm,
	bottom=10mm,
	left=25mm,
	right=25mm,
	headsep=1.5cm,
	includehead,
}
\RequirePackage{netcommon}

% Alle Konfigurationsbefehle sind optional. Fehlende Befehle fueheren einfach
% zu "blank forms".

% Typ der Arbeit/Einstellung. Gueltige Argumente sind:
% bachelor,master,diplom,idp,gr,hiwi,other
% Falls 'other' gewaehlt wird, kann als optionales Argument eine spezielle Art
% von Abschlussarbeit angegeben werden, z.B. \type[Sklave]{other}. Andernfalls
% wird 'Other' als Standardbeschreibung gesetzt.
\type{bachelor}

% Informationen ueber den Studenten. Sollte selbsterklaerend sein.
\anrede{Herr}
\nachname{Sternsdorf}
\vorname{Simon}
\matrikel{3652595}
\sunhalle{sternsdo}
\semester{6}{SoSe\,2017}
\studientelefon{}{tel}
\heimattelefon{}{--}
\studienadresse{strasse}{plz stadt}
\heimatadresse[adresszusatz=,appartment=]{}{}

% Informationen ueber die Arbeit. Sollte selbsterklaerend sein.
\themensteller{\NEThead}
\beginn{04}{2017}
\endt{08}{2017}
\betreuer{Florian Wohlfahrt}
\title{Performance
Analysis of Middlebox Functionality}{Leistungsanalyse der Funktionen von Middleboxes}
\studiengang{Informatik}


% Falls \type{hiwi} gesetzt wurde, wird die Taetigkeit auf dem Aufnahmeformular
% des Lehrstuhls angegeben.
\taetigkeit{test}

\pagestyle{scrheadings}
\chead{\TUMheader{1cm}}

\renewcommand{\maketitle}{%
	\begin{center}
		\textbf{\introductoryheadline}%

		\Large%
		\textbf{\thetitle}%
	\end{center}

	\footnotesize%
	\hrule
	\vskip1ex
	\begin{tabular}{ll}
		\thenamelabel: & \thevorname{} \textbf{\thenachname}\\
		\theadvisorlabel: & \hspace*{-.5ex}\thebetreuer\\
		\thesupervisorlabel: & \chairhead\\
		\thebeginlabel: & \thebeginnmonat/\thebeginnjahr\\
		\theendlabel: & \theendmonat/\theendjahr\\
	\end{tabular}
	\vskip1ex
	\hrule
	\vskip4ex
}

\linespread{1.2}
\setlength{\parskip}{.5\baselineskip}

\begin{document}
\maketitle

\subsection*{Topic}
This Bachelor's Thesis shall implement a wide band scanning application
resembling a spectrum analyzer using software defined radio hardware.
It shall be possible to both sweep the whole spectrum supported by the hardware
in use as well as tuning into specific parts of that spectrum.
In addition it shall be possible to record a narrow band of the spectrum for
both offline-analyses and replay at a later time.

The intended field of application is, for instance, the generation of
reproducible error signals in a testbed or empirically measuring the gain
of different antennas.

The thesis is thematically similar to the Master's Thesis \emph{Automatic Line
Code Detection of RF Signals on Software Defined Radios} \cite{riedl2015} by
Martin Riedel, which also implemented a spectrum analyzer.
However, it was never planned to record or replay signals.
In addition, this thesis lacks some scientific rigor with respect to the
validity of recorded samples.


\subsection*{Approach}
The thesis is split in two parts.
In the first part we will make measurements and statistic evaluations to
ensure, that the acquired data, which comes from the SDR, is valid.  Therefore
we will test, if the samples are valid right after the internal tuning phase of
the SDR.
We will also investigate whether or not external influences such as the USB
interface interfere with the process.
Other influence such as the quantization error will also be analyzed.
The latter are of particular interest when it comes to replay of previously
recorded signals.

The second part is about the scanning application itself.
The program can use the wide range of the present USRP hardware, which covers a
frequency range from 75\,MHz to 6\,GHz.
The tool would scan the whole band for online investigation or store the
samples for later use.
Further more, you will be able to rescan narrow bands of the spectrum to have a
closer look whats happening there.
The UI will consist of a zoomable heatmap (waterfall diagram) interface with
mouse support.

To get proper values to display, we have to consider different parameters like
integration time and gain settings etc.
These parameters will also be part of the research.

The software will be based on the GNUradio \cite{gnuradio} software defined radio
framework.  The additional work is planed to be done with python3.
The used hardware consists of 2 URSP b210, where for the main part only one is
needed.


\subsection*{Previous work}
I did some previous work with SDR hardware, the RTL-SDR \cite{rtlsdr} and the
hackrf \cite{hackrf}.  I am quite familiar with the software (GNUradio) and
hardware environment.
After talking to Stephan Günther about a possible thesis, I set up a demo
\cite{gnuradiodemo}\cite{gnuradiodemo2} for modulation schemes in 802.11 a/g/p,
which could be used in the lecture.
The demo shows a scope plot / constellation form for a simulated 802.11
transmission.
When playing with the mutable parameters like signal-to-noise-ration, PDU
ratio, PDU size and the actual modulation schemes (BPSK, QPSK, 16-QAM, 64-QAM)
you can see how they influence the frame error rate.
This can be observed by taking a look at the constellation diagram or the
detected frame error rate itself.

\subsection*{Additional hardware requirements}
\begin{itemize}
	\item 2 chassis for the USRPs \cite{chassiusrp}
\end{itemize}

\bibliographystyle{IEEEtran}
\bibliography{IEEEabrv,lit}


\end{document}
