\documentclass[german,11pt]{netforms}

\usepackage[T1]{fontenc}
\usepackage[utf8]{inputenc}
\usepackage{geometry}
\usepackage{booktabs}

\RequirePackage{netcommon}

% Alle Konfigurationsbefehle sind optional. Fehlende Befehle fueheren einfach
% zu "blank forms".

% Typ der Arbeit/Einstellung. Gueltige Argumente sind:
% bachelor,master,diplom,idp,gr,hiwi,other
% Falls 'other' gewaehlt wird, kann als optionales Argument eine spezielle Art
% von Abschlussarbeit angegeben werden, z.B. \type[Sklave]{other}. Andernfalls
% wird 'Other' als Standardbeschreibung gesetzt.
\type{bachelor}

% Informationen ueber den Studenten. Sollte selbsterklaerend sein.
\anrede{Herr}
\nachname{Sternsdorf}
\vorname{Simon}
\matrikel{3652595}
\sunhalle{sternsdo}
\semester{6}{SoSe\,2017}
\studientelefon{}{tel}
\heimattelefon{}{--}
\studienadresse{strasse}{plz stadt}
\heimatadresse[adresszusatz=,appartment=]{}{}

% Informationen ueber die Arbeit. Sollte selbsterklaerend sein.
\themensteller{\NEThead}
\beginn{04}{2017}
\endt{08}{2017}
\betreuer{Florian Wohlfahrt}
\title{Performance
Analysis of Middlebox Functionality}{Leistungsanalyse der Funktionen von Middleboxes}
\studiengang{Informatik}


% Falls \type{hiwi} gesetzt wurde, wird die Taetigkeit auf dem Aufnahmeformular
% des Lehrstuhls angegeben.
\taetigkeit{test}

\newgeometry{
	top=2.5cm,
	bottom=2.5cm,
	left=2cm,
	right=2cm,
}

\makeatletter
	\addtolength{\@totalleftmargin}{-1em}%
	\def\makecell#1#2{%
		\begin{minipage}[c][2ex]{#1}%
			\mbox{}#2
		\end{minipage}
	}


\newcommand{\anwendungsfach}[1]{%
	\def\theanwendungsfach{#1}
}

\renewcommand{\title}[2]{%
	\def\thetitlegerman{#1}
	\def\thetitleenglish{#1}
}

\newcommand{\supervisor}[1]{%
	\def\thesupervisor{#1}
}

\newcommand{\anzahlbearbeiter}[1]{%
	\def\theanzahlbearbeiter{#1}
}

\newcommand{\idplecture}[4]{%
	\def\theidpmodule{#1}
	\def\theidplecture{#2}
	\def\theidplecturer{#3}
	\def\theidpects{#4}
}
\newcommand{\idplecturesum}[1]{%
	\def\theidplecturesum{#1}
}

\newcommand{\idpects}[4]{%
	\def\theidpectslecturesi{#1}
	\def\theidpectspracticali{#2}
	\def\theidpectsdocumentationi{#3}
	\def\theidpectspresentationi{#4}
}

\newcommand{\idpbearbeiteri}[4]{%
	\def\theidpbearbeitermatrikeli{#1}
	\def\theidpbearbeiterlastnamei{#2}
	\def\theidpbearbeiterfirstnamei{#3}
	\def\theidpbearbeitersemesteri{#4}
}
\newcommand{\idpbearbeiterii}[4]{%
	\def\theidpbearbeitermatrikelii{#1}
	\def\theidpbearbeiterlastnameii{#2}
	\def\theidpbearbeiterfirstnameii{#3}
	\def\theidpbearbeitersemesterii{#4}
}
\newcommand{\idpstart}[1]{%
	\def\theidpstart{#1}
}
\newcommand{\idpend}[1]{%
	\def\theidpend{#1}
}

\newcounter{tumtutorial@assistants@cnt}
\newcommand{\assistants}[1]{%
	\def\tumtutorial@assistants{#1}
	\setcounter{tumtutorial@assistants@cnt}{0}
	\foreach \a in \tumtutorial@assistants {%
		\stepcounter{tumtutorial@assistants@cnt}
		\expandafter\expandafter\expandafter\xdef\expandafter\csname tumtutorial@assistant\thetumtutorial@assistants@cnt\endcsname{\a}
	}
}
\newcommand{\theassistants}[1][]{%
	\ifx\@empty#1\@empty
		\tumtutorial@assistants
	\else%
		\csname tumtutorial@assistant#1\endcsname
	\fi%
}

\newcommand{\assistantsmail}[1]{%
	\def\tumtutorial@assistantsmail{#1}
	\setcounter{tumtutorial@assistants@cnt}{0}
	\foreach \a in \tumtutorial@assistantsmail {%
		\stepcounter{tumtutorial@assistants@cnt}
		\expandafter\expandafter\expandafter\xdef\expandafter\csname tumtutorial@assistantmail\thetumtutorial@assistants@cnt\endcsname{\a}
	}
}
\newcommand{\theassistantsmail}[1][]{%
	\ifx\@empty#1\@empty
		\tumtutorial@assistantsmail
	\else%
		\csname tumtutorial@assistantmail#1\endcsname
	\fi%
}
\makeatother

\anwendungsfach{Elektrotechnik}
\title{TUMexam}{TUMexam}
\supervisor{Prof.~Dr.-Ing.~Wolfgang Utschick}
\assistants{Dr.-Ing.~Michael Joham, Dipl.~Ing.~Stephan~M.~Günther}
\anzahlbearbeiter{2}
\idplecture{EI0632}{Mesch-Maschine-Kommunikation 1}{Prof.~Dr.-Ing.~Rigoll}{5}
\idplecturesum{5}
\idpects{5}{8}{2}{1}
\idpbearbeiteri{03638631}{Leppelsack}{Hendrik}{1}
\idpbearbeiterii{03627127}{Jaeger}{Benedikt}{3}
\idpstart{01.10.2016}
\idpend{30.03.2017}

\begin{document}
	\begin{minipage}{.2\textwidth}%
		\centering
		\INlogo{height=1.2cm,color=TUMDarkerBlue}%
	\end{minipage}%
	\begin{minipage}{.6\textwidth}%
		\centering
		\huge\textsc{FAKULT\"AT F\"UR INFORMATIK}\large\\[-.2ex]
		\textsc{DER TECHNISCHEN UNIVERSI\"AT M\"UNCHEN}\\[-.2ex]
		\textsc{PR\"UFUNGSAUSSCHUSS}\\[-.2ex]
		Vorsitzender: Univ.~Prof.~Bernd~Br\"ugge, Ph\,D.
	\end{minipage}%
	\begin{minipage}{.2\textwidth}%
		\centering
		\TUMlogo{shape=outline,height=1.5cm,color=TUMDarkerBlue}%
	\end{minipage}\\[1ex]

	\begin{center}
		\textbf{%
			\large
			\underline{Antrag f\"ur ein Interdisziplin\"ares Projekt f\"ur
			Studierende des Studiengangs Master}\\[.2ex]
			\underline{Informatik\,/\,Anmeldung der Bearbeiter/innen}
		}
	\end{center}

	\vspace{1ex}
	{%
		\footnotesize
		\underline{Bitte das auf Seite~2 unterschriebene Formular
		zur\"ucksenden an:}
	}

	\vspace{.5ex}
	\hspace{.5ex}
	\fbox{
		\begin{minipage}[t][2.5cm]{9cm}
			An den Vorsitzenden des Pr\"ufungsausschusses\\
			Herrn Prof.~Bernd Br\"ugge, Ph.\,D.\\
			Fakult\"at f\"ur Informatik der TUM\\
			Boltzmannstra\ss{}e 3\\
			D-85748 Garching bei M\"unchen
		\end{minipage}
	}

	\vspace{\baselineskip}
	Falls das Projekt bereits genehmigt wurde und nur Bearbeiter/innen
	angemeldet werden, ist die Genehmigung des Projekts in Kopie beizulegen;
	Punkt~1 muss dann nicht nochmals ausgef\"ullt werden.

	\vspace{\baselineskip}
	{\large\textbf{1.\, Antrag f\"ur ein Interdisziplin\"ares Projekt}}\\
	\begin{tabular}{p{8em}l}
		Anwendungsfach:%
		&
		\fbox{%
			\begin{minipage}[c][3ex]{12.8cm}
			\mbox{\theanwendungsfach}
			\end{minipage}
		}\\[1.2ex]

		Thema (deutsch \textbf{und} englisch):%
		&
		\fbox{%
			\begin{minipage}[c][9ex]{12.8cm}
			\mbox{\thetitlegerman}

			\mbox{\thetitleenglish}
			\end{minipage}
		}\\[4.3ex]

		Aufgabensteller/in:%
		&
		\fbox{%
			\begin{minipage}[c][3ex]{12.8cm}
			\mbox{\thesupervisor}
			\end{minipage}
		}\\[1.3ex]

		Betreuer/in:%
		&
		\fbox{%
			\begin{minipage}[c][3ex]{12.8cm}
			\mbox{\theassistants}
			\end{minipage}
		}
	\end{tabular}

	\vspace{.5\baselineskip}

	\begin{enumerate}\itemsep-2pt
		\item[a)] Kurzbeschreibung des Projekts: mindestens 2-seitige Beschreibung
		als \textbf{Anlage}.\\
		{\footnotesize Sie sollte die Infrmatik- als auch die Anwendungsfachanteile
		beinhalten, die zu bearbeitenden Meilensteine im Rahmen des Projekts
		grob skizzieren und aufzeigen, wo der Vorlesungsinhalt f\"ur die
		Projektbearbeitung n\"otig ist.}
		\item[b)] Vorgesehene Anzahl der Bearbeiter/innen:\hspace{1ex}%
		\fbox{%
			\begin{minipage}[c][3ex]{1cm}
				\hfil\mbox{\theanzahlbearbeiter}\hfill
			\end{minipage}
		}

		{\footnotesize Bei mehreren Bearbeitern ist in der Projektbeschreibung
		die Aufgabenteilung anzugeben.}
		\item[c)] Vorbereitende\,/\,begleitende Vorlesungen (im Umfang von
		mindestens 5\,ECTS):

		\renewcommand{\arraystretch}{1.5}
		\setlength\tabcolsep{.4ex}
		\begin{tabular}{|l|l|l|l|}
			\hline
			\textbf{Modulnr}
				& \textbf{Vorlesungen}
				& \textbf{Dozent}
				& \textbf{ECTS}\\
			\hline
			\fbox{\makecell{.075\textwidth}{\theidpmodule}}
				& \fbox{\makecell{.5\textwidth}{\theidplecture}}
				& \fbox{\makecell{.2\textwidth}{\theidplecturer}}
				& \fbox{\makecell{.06\textwidth}{\hfill\theidpects\hfill}}
				\\[0.5ex]
			\hline
			\fbox{\makecell{.075\textwidth}{}}
				& \fbox{\makecell{.5\textwidth}{}}
				& \fbox{\makecell{.2\textwidth}{}}
				& \fbox{\makecell{.06\textwidth}{\hfill\hfill}}
				\\[0.5ex]
			\hline
			\fbox{\makecell{.075\textwidth}{}}
				& \fbox{\makecell{.5\textwidth}{}}
				& \fbox{\makecell{.2\textwidth}{}}
				& \fbox{\makecell{.06\textwidth}{\hfill\hfill}}
				\\[0.5ex]
			\hline
				& \multicolumn{2}{|r|}{\makecell{.37\textwidth}{\textbf{Summe ECTS
				(mindestens 5\,ECTS):}}}
				& \fbox{\makecell{.06\textwidth}{\hfill\theidplecturesum\hfill}}
				\\[0.5ex]
			\hline
		\end{tabular}
	\end{enumerate}

	\clearpage
	\begin{enumerate}
		\item[d)] Benotung:

		\renewcommand{\arraystretch}{1.5}
		\setlength\tabcolsep{.4ex}
		\begin{tabular}{|p{.85\textwidth}|l|}
			\hline
			\textbf{Gewichte zur Festsetzung der Gesamtnote aus den Einzelnoten}
				& \hfil\textbf{ECTS}\hfill\\
			\hline
				ECTS Vorlesungen gesamt (siehe Vorderseite, mindestens 5\,ECTS)
				&
				\fbox{\makecell{.075\textwidth}{\hfill\theidpectslecturesi\hfill}}
				\\[0.5ex]
			\hline
				ECTS Praktische T\"atigkeit
				&
				\fbox{\makecell{.075\textwidth}{\hfill\theidpectspracticali\hfill}}
				\\[0.5ex]
			\hline
				ECTS Dokumentation (mindestens 2\,ECTS)
				&
				\fbox{\makecell{.075\textwidth}{\hfill\theidpectsdocumentationi\hfill}}
				\\[0.5ex]
			\hline
				ECTS Pr\"asentation (mindestens 1\,ECTS)
				&
				\fbox{\makecell{.075\textwidth}{\hfill\theidpectspresentationi\hfill}}
				\\[0.5ex]
			\hline
				\textbf{ECTS insgesamt}
				& \fbox{\makecell{.075\textwidth}{\hfill\textbf{16}\hfill}}
				\\[0.5ex]
			\hline
		\end{tabular}

		{\footnotesize
		\textbf{Hinweis:}\\
		1\,ECTS-Punkt entspricht 30 Arbeitsstunden}
	\end{enumerate}


	\textbf{2.\,Anmeldung der Bearbeiter/innen}

	\vspace{.5\baselineskip}
	Bearbeiter/innen:

	\vspace{.3\baselineskip}
	\renewcommand{\arraystretch}{1.5}
	\setlength\tabcolsep{.4ex}
	\begin{tabular}{|l|l|l|p{.26\textwidth}|}
		\hline
			\small \hfil Matrikelnummer\hfill
			& \small \hfil Name, Vorname\hfill
			& \small \hfil Semester\hfill
			& \small \hfil Unterschrift\hfill\\
		\hline
			\fbox{\makecell{.12\textwidth}{\theidpbearbeitermatrikeli}}
			& \fbox{\makecell{.45\textwidth}{\theidpbearbeiterlastnamei,
			\theidpbearbeiterfirstnamei}}
			&
			\fbox{\makecell{.07\textwidth}{\hfill\theidpbearbeitersemesteri\hfill}}
			&
			\\[0.5ex]
		\hline
			\fbox{\makecell{.12\textwidth}{\theidpbearbeitermatrikelii}}
			& \fbox{\makecell{.45\textwidth}{\theidpbearbeiterlastnameii,
			\theidpbearbeiterfirstnameii}}
			&
			\fbox{\makecell{.07\textwidth}{\hfill\theidpbearbeitersemesterii\hfill}}
			&
			\\[0.5ex]
		\hline
			\fbox{\makecell{.12\textwidth}{}}
			& \fbox{\makecell{.45\textwidth}{}}
			& \fbox{\makecell{.07\textwidth}{\hfill\hfill}}
			&
			\\[0.5ex]
		\hline
			\fbox{\makecell{.12\textwidth}{}}
			& \fbox{\makecell{.45\textwidth}{}}
			& \fbox{\makecell{.07\textwidth}{\hfill\hfill}}
			&
			\\[0.5ex]
		\hline
			\fbox{\makecell{.12\textwidth}{}}
			& \fbox{\makecell{.45\textwidth}{}}
			& \fbox{\makecell{.07\textwidth}{\hfill\hfill}}
			&
			\\[0.5ex]
		\hline
	\end{tabular}

	\renewcommand{\arraystretch}{2}
	\begin{tabular}{ll}
		Beginn des Projekts:
			& \fbox{\makecell{.3\textwidth}{\theidpstart}}\\
		Voraussichtlicher Abschluss: \ 
			& \fbox{\makecell{.3\textwidth}{\theidpend}}
	\end{tabular}

	\vspace{2\baselineskip}
	\setlength{\fboxrule}{3pt}%
	\fbox{%
		\begin{minipage}[c][7em]{.97\textwidth}
			\begin{tabular}{lp{1ex}l}
				\setlength{\fboxrule}{.5pt}%
				\fbox{\makecell{.45\textwidth}{}}
					&
					& \rule[-1ex]{.48\textwidth}{.5pt}\\
				(Ort und Datum)
					&
					& (Unterschrift des\,/\,der Aufgabenstellers/in)
			\end{tabular}
		\end{minipage}
	}

	\vspace{\baselineskip}
	{
		\footnotesize
		\textbf{Hinweis:}\\
		Da das Interdisziplin\"are Projekt Bestandteil der Masterpr\"ufung ist,
		muss der Aufgabensteller f\"ur das Anwendungsfach pr\"ufungsberechtigt
		sein.
		Trifft dies auf den\,/\,die Aufgabensteller/in nicht zu, so ist eine
		besondere Genehmigung durch den Pr\"ufungsausschuss der Fakult\"at
		f\"ur Informatik n\"otig, bzw.\ ein/e geeignete/r Aufgabensteller/in
		anzugeben.
	}
\end{document}
