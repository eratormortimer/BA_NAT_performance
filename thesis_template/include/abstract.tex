\begin{abstract}
	Network Address Translation (NAT) is one of the most common middleboxes deployed by both home-users and End-user Internet Service Providers (ISPs). These middleboxes are still mostly implemented in hardware. These middleboxes have many disadvantages to software middleboxes deployed on commodity hardware. They are less customizable and more expensive because of the dedicated hardware they use. On the other hand they have a better performance for the specific task the hardware was created for and are more energy efficient. To improve this, software middleboxes get further developed and new features are added to speed up the packet processing. $\newline \newline$
In this thesis we operate a testbed to measure the performance of packet processing of specific middlebox software. We also try to localize the performance bottlenecks of packet processing tasks. We will focus on NAT as a simple and widespread example of middleboxes.
\end{abstract}
\cleardoublepage

\begin{otherlanguage}{ngerman}
\begin{abstract}
   Network Address Translation (NAT) ist eine der am meisten gebräuchlichen Middleboxes die sowohl in normalen Haushalten wie auch von End-user Internet Service Providern (ISPs) eingesetzt werden. Diese Middleboxes werden meistens in Hardware implementiert. Sie haben viele Nachteile gegenüber in Software implementierten Middleboxen die auf handelsüblicher Hardware laufen. Man kann sie nicht so gut an genaue Gegebenheiten anpassen und sie sind meist teurer wegen ihrer speziellen Hardware. Auf der anderen Seite sind sie aber performanter für ihren speziellen Anwendungszweck und sind energie-effizienter. Um dies in Zukunft zu verbessen werden die Software Middleboxen immer weiter entwickelt und bekommen neue Features um die Packetverarbeitung zu beschleunigen. $\newline \newline$
In dieser Bachelorarbeit werden wir in einem testbed arbeiten um die Leistung von bestimmten Software Middleboxes bei der Packetverarbeitung zu messen. Wir werden außerdem versuchen die einschränkenden Faktoren zu lokalisieren die die Leistung der Middlebox beim Packetverarbeiten verringern. Dabei werden wir uns auf NAT konzentrieren da es ein weit verbreitetes Beispiel einer Middlebox ist. 
   
   
   
   
\end{abstract}
\end{otherlanguage}
\cleardoublepage


