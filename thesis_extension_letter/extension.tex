\documentclass[a4paper,titlepage,10pt]{article} 

\usepackage{geometry}
\usepackage{fancyhdr}
\usepackage[german]{netcommon}
\usepackage[utf8]{inputenc}
\usepackage[T1]{fontenc}
\usepackage{helvet}
\usepackage[ngerman]{babel}

% font settings
\renewcommand*\familydefault{\sfdefault}

% page design settings
\geometry{a4paper, left=25mm,right=25mm, top=3.5cm, bottom=2cm} 
\pagenumbering{gobble}
\parindent0pt
\pagestyle{fancy}

% page header
\headheight24pt
\lhead{\scriptsize Antrag an den\\ Prüfungsausschus für Informatiker}
\rhead{\scriptsize Studiensekretariat Informatik\\Raum: 00.10.041\\Telefon: 17558, Fax: 17559}

% In the best case only the following information must be changed to get a minimal effort extension letter

% information about student
\newcommand{\name}{Simon Sternsdorf}
\newcommand{\strnum}{Helene-Mayer-Ring 7A}
\newcommand{\zipcity}{80809 München}
\newcommand{\mail}{simon.sternsdorf@tum.de}
\newcommand{\phone}{0176 81515096}
\newcommand{\fieldofstudy}{Informatik (Bachelor)}
\newcommand{\matrikel}{3652595}
\newcommand{\semester}{6}

% stuff about thesis
\newcommand{\reason}{Software probleme}
\newcommand{\dat}{15. Oktober 2017}

\begin{document}
\section*{Antrag auf Verlängerung}
\begin{tabular}{@{}ll}
Antragsteller: & \name{}\\	
Datum: &\today \\
\\
Adresse:&	 \strnum{}\\
			&\zipcity{}\\
E-Mail:		&	\mail{}\\
Telefon:&  \phone{}\\
\\
Studienrichtung:	&\fieldofstudy{}\\
Matrikelnummer: &	\matrikel{}\\
Fachsemester: 	&	\semester{}\\
\end{tabular}
\vspace{3ex}\\
Es wurde bereits eine Verlängerung um einen Monat beantragt und am 15ten August genehmigt.
\vspace{3ex}\\
Ich bitte um eine einmonatige Fristverlängerung zur Abgabe der Bachelorarbeit. Der neue Abgabetermin wäre damit der \dat{}.

$\newline \newline$
Wie bereits in meiner ersten Fristverlängerung ausgeführt wird in dieser Arbeit wurde die Performance von verschiedenen Software-Middleboxes in einem Testbed untersucht.
Wir benutzten mOS für die Middleboxes, ein Framework was schnelle und effiziente Packetverarbeitung ermöglicht. 
Laut Specifikation und Dokumentation war mOS genau für die Aufgabe geeignet. 
Leider bemerkten wir im Testbetrieb -- also nachdem wir einigen Aufwand
in die Implementierung gesteckt hatten -- dass die mOS Software einen
schwerwiegenden Fehler aufweist der mOS für unseren Zweck unbrauchbar macht. mOS
ist nicht in der Lage mehr als 23 Paketströme (flows) gleichzeitig zu
verarbeiten. Alle Versuche diesen Fehler zu reparieren, auch mit Hilfe der Developer der Software, schlugen leider fehl. 

$\newline \newline$
Wegen dieser Probleme musste auf eine neue Software umgestellt werden was die Arbeit stark verzögert hat. 
Zum Zeitpunkt der Beantragung der ersten Fristverlängerung habe ich (zusammen mit meinem Betreuer) den Zeitverlust noch zu gering eingeschätzt. Zum heutigen Zeitpunkt ist klar, dass sich die Bearbeitung meiner Arbeit wegen der genannten Probleme um mehr als den bereits beantragten Monat verzögert hat. Deshalb beantrage ich die oben genannte Fristverlängerung um die Zeitverluste wegen der Softwareprobleme auszugleichen.

\vskip4em
Garching b. München, 
\rule[-.4em]{2cm}{0.25mm}
\hfill%
	\rule[-.4em]{7cm}{0.25mm}

\vskip.3em \hfill  \name{}

\vfill
\textbf{Kurze Stellungnahme und Unterschrift des Themenstellers}

\vskip2em
\hspace{.5cm}\rule[-.4em]{15cm}{0.25mm}
%\vskip2em
%\hspace{.5cm}\rule[-.4em]{15cm}{0.25mm}

\vskip3em
Garching b. München, 
\rule[-.4em]{2cm}{0.25mm}
\hfill%
\rule[-.4em]{7cm}{0.25mm}

\vskip.3em \hfill  \NEThead{}

\end{document}
