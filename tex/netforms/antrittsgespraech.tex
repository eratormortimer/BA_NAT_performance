\documentclass[NET,a4,12pt,ngerman]{netforms}

\usepackage[utf8]{inputenc}
\usepackage{tumlang}
\usepackage{tumcontact}
\usepackage{scrpage2}

\geometry{%
	top=10mm,
	bottom=10mm,
	left=25mm,
	right=25mm,
	headsep=1.5cm,
	includehead,
}
\RequirePackage{netcommon}

% Alle Konfigurationsbefehle sind optional. Fehlende Befehle fueheren einfach
% zu "blank forms".

% Typ der Arbeit/Einstellung. Gueltige Argumente sind:
% bachelor,master,diplom,idp,gr,hiwi,other
% Falls 'other' gewaehlt wird, kann als optionales Argument eine spezielle Art
% von Abschlussarbeit angegeben werden, z.B. \type[Sklave]{other}. Andernfalls
% wird 'Other' als Standardbeschreibung gesetzt.
\type{bachelor}

% Informationen ueber den Studenten. Sollte selbsterklaerend sein.
\anrede{Herr}
\nachname{Sternsdorf}
\vorname{Simon}
\matrikel{3652595}
\sunhalle{sternsdo}
\semester{6}{SoSe\,2017}
\studientelefon{}{tel}
\heimattelefon{}{--}
\studienadresse{strasse}{plz stadt}
\heimatadresse[adresszusatz=,appartment=]{}{}

% Informationen ueber die Arbeit. Sollte selbsterklaerend sein.
\themensteller{\NEThead}
\beginn{04}{2017}
\endt{08}{2017}
\betreuer{Florian Wohlfahrt}
\title{Performance
Analysis of Middlebox Functionality}{Leistungsanalyse der Funktionen von Middleboxes}
\studiengang{Informatik}


% Falls \type{hiwi} gesetzt wurde, wird die Taetigkeit auf dem Aufnahmeformular
% des Lehrstuhls angegeben.
\taetigkeit{test}

\pagestyle{scrheadings}
\chead{\TUMheader{1cm}}

\renewcommand{\maketitle}{%
	\begin{center}
		\textbf{\introductoryheadline}%

		\Large%
		\textbf{\thetitle}%
	\end{center}

	\footnotesize%
	\hrule
	\vskip1ex
	\begin{tabular}{ll}
		\thenamelabel: & \thevorname{} \textbf{\thenachname}\\
		\theadvisorlabel: & \hspace*{-.5ex}\thebetreuer\\
		\thesupervisorlabel: & \chairhead\\
		\thebeginlabel: & \thebeginnmonat/\thebeginnjahr\\
		\theendlabel: & \theendmonat/\theendjahr\\
	\end{tabular}
	\vskip1ex
	\hrule
	\vskip4ex
}

\linespread{1.2}
\setlength{\parskip}{.5\baselineskip}

\begin{document}
\maketitle

\subsection*{Topic}
In this Bachelor's Thesis we test different Implementations of Middleboxes. 
We will set up different Middleboxes in our testbed and then measure the performance of packet-processing in our testbed. 
Our advanced goal is to identify different bottlenecks in modern Middlebox implementations. As a common Middlebox we will quantify NAT-boxes. 

The intended field of application is, for instance, in big ISP networks. ISP's often use middleboxes in there setups to cache or even modify user data.
NAT-boxes play a huge role in cellular networks, where ISP's use them to give private IP's to mobile devices. Performance improvements of even milliseconds in these Middleboxes can be very impactful in such huge networks. 

The thesis is thematically similar to the Bachelor's Thesis \emph{A Model for Performance Prediction
in PC-based Packet Processing Systems} \cite{riedl2015} by
Dominik Scholz, where he tested the general performance of the Linux Network Stack and generated a performance prediction model for some use cases. 
He did not directly test Middlebox implementations and especially no NAT implementations. We can build on his thesis and will use similar methods to test Middleboxes in our testbed. 


\subsection*{Approach}
The thesis is split in two parts.
In the first part we will make measurements and statistic evaluations to
ensure, that the acquired data, which comes from the SDR, is valid.  Therefore
we will test, if the samples are valid right after the internal tuning phase of
the SDR.
We will also investigate whether or not external influences such as the USB
interface interfere with the process.
Other influence such as the quantization error will also be analyzed.
The latter are of particular interest when it comes to replay of previously
recorded signals.

The second part is about the scanning application itself.
The program can use the wide range of the present USRP hardware, which covers a
frequency range from 75\,MHz to 6\,GHz.
The tool would scan the whole band for online investigation or store the
samples for later use.
Further more, you will be able to rescan narrow bands of the spectrum to have a
closer look whats happening there.
The UI will consist of a zoomable heatmap (waterfall diagram) interface with
mouse support.

To get proper values to display, we have to consider different parameters like
integration time and gain settings etc.
These parameters will also be part of the research.

The software will be based on the GNUradio \cite{gnuradio} software defined radio
framework.  The additional work is planed to be done with python3.
The used hardware consists of 2 URSP b210, where for the main part only one is
needed.


\subsection*{Previous work}
I participated in and passed the iLab1 Bachelor course. There I leaned a lot about networking in general and also worked a lot with the tools necessary to operate the testbed. I also picked up shell scripting and python during the Network Security lecture.  


\bibliographystyle{IEEEtran}
\bibliography{IEEEabrv,lit}


\end{document}
